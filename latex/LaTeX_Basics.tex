% Template for LaTeX documents

\title{Notes for Will}  % commands all start with a backslach \
\author{Will Liu}  % braces {} are used for 'arguments'
\date{\today} % commands are case sensitive; \today, not \Today

\documentclass[11pt]{report}  % brackets [] are used for 'optional arguments'
% Classes include: article, report, book, letter
% Options include: sizept (e.g. 11pt), twoside, twocolumn
\usepackage{comment}  % List usepackage after docummentclass

\begin{document} % Basically just needs \begin{document} and \end{document}
\maketitle  % This adds in Title, Author, Date

% Comments using the package 'comment'
\begin{comment}
  This is a
  multi-line comment
  that will be ignored
\end{comment}

\section*{Summary}  % By default there's numbering; you can remove the numbering by adding a '*'
LaTeX has a text mode (default) and a math mode.
Text mode writes plain text.  Math mode lets you output symbols like: $\rho_i$
When in doubt, \em please \em consult the manual!

% How to do a quote, for small quotes and does not indent
\begin{quote}
This is a small quote
\end{quote}

% How to do quotations, used for long quotes and indents
\begin{quotation}
Find a job you love and you'll never work a day in your life.
\end{quotation}

% There's 3 types of lists: itemize, enumerated,
\section*{Lists (Itemize) - Bullets}
% How to list items plainly
Here is a list of plain items
\begin{itemize}  % These are bullets
\item \it How to do italics typeface
\item \sl How to do slanted typeface
\item \bf How to do boldface typeface
\item \sf How to do san serif typeface
\item {\tt How to do typewriter typeface}  % braces limit scope; e.g. you don't have to change font, then change back
\item \em How to do roman or italics typeface
\item \rm How to do normal (roman) typeface  % when you change a font, you must change it back to the default
\item \large How to do larger type
\item \Large How to do even larger type
\item \small How to do small type
\item \normalsize How do to normal size type
\end{itemize}

\section*{Lists (Enumerate) - Numbers and Labels}
% How to list items with numbers or labels
Here is a list of items with numbers and labels
\begin{enumerate}  % These are numbers or labels
\item The enumerate environment numbers the list elements
\item This is the second item
\item[A] This is with a label A
\end{enumerate}

\section*{Lists (Description) - Term and description}
\begin{description}
\item[Badgers] are fierce
\item[Humans] are weak
\item[Robots] are indestructable
\end{description}

% How to use special symbols
\section*{Special Symbols}
Here's how to use special symbols
\begin{itemize}
\item How to use the ampersand \&
\item How to use the left brace \{
\item How to use the right brace \}
\item How to use the dollar sign \$
\item How to use the percent sign \%
\item How to use the number sign \#
\item How to use the underscore \_
\item How to use the hyphen \--
\item How to use the dash \---
\item How to use open quotes \``
\item How to use closed quotes \''
\end{itemize}

\section*{Line Breaks}
\begin{paragraph} Hello Paragraph  \\
This is how to break up lines \\
You just add these little lines here \\
And then your new line starts! \\
Amazing!
\end{paragraph}

\section*{Justification}
\begin{itemize}
\item \begin{center} Center Me \end{center}  % Center Text
\item \begin{flushleft} Place me to the left! \end{flushleft}  % Justify Left
\item \begin{flushright} Place me to the right! \end{flushright}  % Justify Right

\end{itemize}

% Print everything as is
\section*{Verbatim}
\begin{verbatim}
Everything in here is printed as is.
  To position items, use spacing
\end{verbatim}

\section*{Document Parts}
\begin{itemize}
\item Part is a very large part of, say, a book
\item Chapter is a chapter of a book; not in article style
\item Section is a section of the document
\item Subsection is a section then broken into subsections
\item Subsubsection is pieces of a subsection
\item Paragraph contain some paragraphs
\item Subparagraph have even smaller pieces
\end{itemize}


Creating a footnote is easy.\footnote{This is my footnote}

\end{document}
